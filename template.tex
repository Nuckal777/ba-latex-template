\documentclass[ngerman,12pt]{article}

\usepackage[english,main=ngerman]{babel}
\usepackage{float}
\usepackage[autostyle=true,german=quotes]{csquotes}
\usepackage[a4paper,left=3cm,right=2cm,top=2cm,bottom=2cm]{geometry}
%it is recommended to compile with xelatex when using pstricks
\usepackage{pst-uml}
\usepackage[style=authoryear-ibid,minnames=1,maxnames=3,dashed=false,giveninits=true,uniquename=false,uniquelist=false]{biblatex}
\usepackage[nogroupskip,nonumberlist,nopostdot]{glossaries}
\usepackage{wrapfig}
\usepackage{tocbibind}
\usepackage[bottom,hang]{footmisc}
\usepackage[all]{nowidow}
\usepackage{xcolor}
\usepackage{listings}
\usepackage[font={footnotesize}]{caption}
\usepackage{fancyhdr}
\usepackage{tocloft}

\addbibresource{document.bib}
%BA Dresden Citation/Bibliography Style
%According to the style guide a publisher is not required, so formating regarding publisher's is likely wrong
\DeclareNameAlias{labelname}{family-given}
\DeclareNameAlias{sortname}{family-given}
\DeclareFieldFormat{url}{in:\space\url{#1}}
\DeclareFieldFormat{urldate}{\newunitpunct\space(#1)}
\DeclareFieldFormat{year}{#1}
\DeclareFieldFormat{edition}{#1 Auflage}
\DeclareFieldFormat[article]{citetitle}{#1}
\DeclareFieldFormat[article]{title}{\textit{#1}}
\DefineBibliographyStrings{ngerman}{andothers={et al.}}
\DefineBibliographyStrings{ngerman}{ibidem=ebenda}
\setlength{\bibhang}{0pt}
\setlength{\bibitemsep}{1.5\itemsep}
\renewcommand*{\newunitpunct}{\addcomma\space}
\renewcommand*{\labelnamepunct}{\addcolon\space}
\renewcommand*{\subtitlepunct}{\addperiod\space}
\renewcommand*{\multinamedelim}{/}
\renewcommand*{\finalnamedelim}{/}
\renewbibmacro*{textcite:postnote}{%
  \iffieldundef{postnote}
    {\ifbool{cbx:parens}
       {\bibcloseparen}
       {}}
    {\ifbool{cbx:parens}
       {\setunit{\postnotedelim}}
       {\setunit{\extpostnotedelim\bibopenparen}}%
	   \bibcloseparen\printfield{postnote}}}
\providecommand*{\mkibid}[1]{#1}
\renewbibmacro*{cite:ibid}{%
  \printtext{\MakeLowercase{\bibstring{ibidem}}}}
\newbibmacro*{cite:fullpartcite}{%
  \printnames{labelname}%
  \space%
  \bibopenparen%
  \printdateextra%
  \bibcloseparen}

\DeclareCiteCommand{\bacite}[\mkbibfootnote]
  {\usebibmacro{prenote}}
  {\usebibmacro{citeindex}%
   \ifthenelse{\ifciteibid\AND\NOT\iffirstonpage}
     {\usebibmacro{cite:ibid}}
     {\usebibmacro{cite:fullpartcite}}%
	 }
  {\addsemicolon\space}
  {\usebibmacro{postnote}}
\DeclareCiteCommand{\vglcite}[\mkbibfootnote]
  {Vgl. \usebibmacro{prenote}}
  {\usebibmacro{citeindex}%
   \ifthenelse{\ifciteibid\AND\NOT\iffirstonpage}
     {\usebibmacro{cite:ibid}}
     {\usebibmacro{cite:fullpartcite}}%
	 }
  {\addsemicolon\space}
  {\usebibmacro{postnote}}
%End BA Dresden Citation/Bibliography Style
%Frames are required around figures
\newcommand{\bafigure}[2]{
	\begin{figure}[H]
		\fbox{
			\begin{minipage}{\dimexpr\textwidth-2\fboxsep-2\fboxrule\relax}
				\centering
				#2
			\end{minipage}
		}
	\caption[#1]{#1 (Quelle: eigene Darstellung)}
	\label{#1}
	\end{figure}
}
\newcommand{\batable}[2]{
	\begin{table}[H]
		\centering
		#2
	\caption[#1]{#1 (Quelle: eigene Darstellung)}
	\label{#1}
	\end{table}
}

%Table of contents Style
\renewcommand{\cftsecleader}{\cftdotfill{\cftdotsep}} % Linie for chapters
\renewcommand\cftsecdotsep{\cftdot}

%List of figures Style
\renewcommand{\cftfigpresnum}{Abbildung }
\renewcommand{\cftfigaftersnum}{:}
\newlength{\fignumspacing} % a "scratch" length
\settowidth{\fignumspacing}{\bfseries\cftfigpresnum\cftfigaftersnum} % extra space
\addtolength{\cftfignumwidth}{\fignumspacing} % add the extra space

%List of Tables Style
\renewcommand{\cfttabpresnum}{Tabelle }
\renewcommand{\cfttabaftersnum}{:}
\newlength{\tabnumspacing} % a "scratch" length
\settowidth{\tabnumspacing}{\bfseries\cfttabpresnum\cfttabaftersnum} % extra space
\addtolength{\cfttabnumwidth}{\tabnumspacing} % add the extra space

%line spread (1.2 equals roughly 1.5 in word)
\linespread{1.2}

\pagestyle{fancy}
\renewcommand{\headrulewidth}{0pt}
% Clear the header and footer
\fancyhead{}
\fancyfoot{}

% List of Acronym
\setacronymstyle{long-short}
\renewcommand{\glossarysection}[2][]{}
\makenoidxglossaries
% Label Short Long
% Example entry: \newacronym{HTTP}{HTTP}{Hypertext Transfer Protocol}
\bapredoc

\begin{document}
%titlepage
\begin{titlepage}
	\begin{wrapfigure}[3]{r}{0.5\textwidth}
		\begin{center}
			\vspace{-18pt}
			\includegraphics[width=0.48\textwidth]{\baimg}
		\end{center}
	\end{wrapfigure}
	\noindent
	\normalsize
	Berufsakademie Sachsen \\ Staatliche Studienakademie \baplace{} \\ Studiengang \badepartment{} \\
	\par
	\vspace{5cm}
	\begin{center}
		\Huge
		\textbf{\batitle{}}
		\par
		\vspace{4cm}
		\normalsize
		\balastname{}, \bafirstname{} \\
		\bacompany{}\\
		Matrikelnummer: \banumber{}
	\end{center}
	\vfill
	Gutachter: \bacorrector{}\\[\baselineskip]
	Tag der Themenübergabe: \bathemedate{} \\
	Tag der Einreichung: \bareturndate{}
\end{titlepage}
\pagenumbering{Roman}
\addtocounter{page}{1}
\newpage

\subsubsection*{Sperrvermerk}
Die vorliegende Arbeit enthält vertrauliche Daten des Unternehmens \bacompany{}.
Auf Wunsch des Unternehmens \bacompany{} ist die vorliegende Arbeit für die öffentliche Nutzung zu sperren.
Veröffentlichung, Vervielfältigung und Einsichtnahme sind ohne ausdrückliche Genehmigung des Unternehmens \bacompany{}, in \bacompanyplace{} und des Verfassers \bafirstname{} \balastname{} nicht gestattet.
Die Arbeit ist nur den Gutachtern und den Mitgliedern des Prüfungsausschusses zugänglich zu machen.
\bigskip\par
\begin{wrapfigure}[1]{r}{0.3\textwidth}
	\begin{center}
		\vspace{-64pt}
		\includegraphics[width=0.28\textwidth]{\basignature}
	\end{center}
\end{wrapfigure}
\bigskip\bigskip
\par\noindent
\baplace{}, \bareturndate{}
\bigskip
\par\noindent
Ort, Datum\hfill Unterschrift des Verfassers
\newpage

% Set the right side of the footer to be the page number
\fancyfoot[R]{\thepage}

\selectlanguage{english}
\begin{abstract}
	That the abstract.
	\addcontentsline{toc}{section}{\abstractname}
\end{abstract}
\newpage
\selectlanguage{ngerman}

\tableofcontents
\newpage

\listoffigures
\newpage
%List of acronyms
\addcontentsline{toc}{section}{Abkürzungsverzeichnis}
\section*{Abkürzungsverzeichnis}
\printnoidxglossaries
\newpage

\listoftables
\newpage

%Spacing after paragraphs
\setlength{\parskip}{\baselineskip}
\pagenumbering{arabic}
\setcounter{page}{1}

%content
\bacontent

%Bibliography
\linespread{1.0}
\printbibliography[heading=bibintoc]
\newpage
\fancyfoot{}

\linespread{1.2}
\subsubsection*{Eidesstattliche Erklärung}
Ich erkläre an Eides statt, dass ich die vorliegende Arbeit selbständig und ohne unerlaubte fremde Hilfe angefertigt, andere als die angegebenen Quellen und Hilfsmittel nicht benutzt habe.
Die aus fremden Quellen direkt oder indirekt übernommenen Stellen sind als solche kenntlich gemacht.
Die Zustimmung des Partnerunternehmens in der Praxis zur Verwendung betrieblicher Unterlagen habe ich eingeholt.
Die Arbeit wurde bisher in gleicher oder ähnlicher Form keiner anderen Prüfungsbehörde vorgelegt und auch nicht veröffentlicht.
\par
\begin{wrapfigure}[1]{r}{0.3\textwidth}
	\begin{center}
		\vspace{-64pt}

		\includegraphics[width=0.28\textwidth]{\basignature}
	\end{center}
\end{wrapfigure}
\bigskip
\par\noindent
\baplace{}, \bareturndate{}
\par\noindent
Ort, Datum\hfill Unterschrift des Verfassers

\end{document}